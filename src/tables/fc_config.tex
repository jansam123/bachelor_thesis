\begin{table}[h]
    \centering
    \caption{Configuration of FC and Highway models.}
    \label{tab:fc_config}
    \begin{tabular}{@{}lcc@{}}
    \toprule
    \textbf{Parameter} & \textbf{Fully Connected} & \textbf{Highway} \\ \midrule
    Layer Size & 512 & 344 \\
    Number of Layers & 11 & 11 \\
    Dropout & 0.2 & 0.2 \\
    Activation & Swish & Swish \\
    \bottomrule
    \end{tabular}
\end{table}