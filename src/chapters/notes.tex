\chapwithtoc{Notes}

\section{12.4.2023}

\subsection{Introduction}
\begin{itemize} 
\item To detect -> To produce
\item chapter chapter -> chapter
\item chapter chapter -> chapter
\item "making graphics designers obsolete" ...no, jestli to tam fakt chces 
mit :D
\item mozna do uvodu pridej jeste jeden odstavecek o tom, k cemu vsemu by se 
dal q/g tagging pouzit \green{Nieco som pridal, ale asi by som potreboval pomoct, ze ktore konkretne analyzy to pouzivaju.}
\end{itemize}

\subsection{Chap 1}
\begin{itemize} 
\item "All quarks can interact weakly and electromagnetically via the Strong 
Force" :)
\item p.8 dole - pozor! Vektorove bosony, ktere uvadis, jsou zaroven 
kalibracni bosony!!! Jestli chces i na tomto miste udelat deleni do 
skupin, tak by asi bylo nejlepsi nazvat je gauge bosons a ta druha 
skupina by nemela nazev a byl by v ni pouze higgs :) Ja osobne bych v 
tomto miste deleni do skupin nedelal.
\item p.9 "Gluons are also massless, making the Strong force also infinite." 
tohle bych urcite odstranil. Sice pak vysvetlujes, ze naboj gluonu 
situaci dramaticky meni, ale stejne by tahle veta podle me nemela 
existovat. Naopak nekde zdurazni, ze silna interakce ma velice kratky dosah
\item "chargers" -> "charges"
\item p.13 nahore: "psi(x)"
\item "In QED , we have the asymptotic freedom, which means the coupling 
constant g goes to zero as the distance between the particles goes to 
infinity." ...tohle je spatne! Pod pojmem "asymptotic freedom" se 
rozumi, kdyz sila interakce jde k nule pro *velmi blizke* castice, coz 
je to same jako *vysoce energeticke* srazky. QED *neni* asymptoticky 
volna. Asymptoticky volna je QCD.
\green{Ja som to vzdy chapal naopak, ze ked asymptoticky idu castic do nekonecna od seba tak su volne}
\end{itemize}

\subsection{Chap 2}
\begin{itemize} 
\item p.28 "which is the highest luminosity of any accelerator in the world" 
...tohle neni pravda: SuperKEKB ma vyssi luminozitu: "On 15 June 2020, 
the SuperKEKB reached an instantaneous luminosity of 2.22x1034 cm-2s-1 — 
surpassing the LHC's record of 2.14x1034 cm-2s-1 set with proton-proton 
collisions in 2018. A few days later, SuperKEKB pushed the luminosity 
record to 2.4x1034 cm-2s-1.[18] In June 2022 the luminosity record was 
nearly doubled to 4.7x1034 cm-2s-1.[19]"
\green{To je pre mna novinka :)}
\item p.30 "providing the most precise tracking system with a precision of 
up to 25 µm" ...tady mi nejak nesedi to "the most precise". Jednak u 
pixeloveho detektoru uvadis hodnotu 10 µm a za druhe je, myslim si, 
lepsi posuzovat presnost ATLAS trackeru jako celku a nikoliv jeho 
jednotlivych casti.
\item p.34 "The high luminosity that LHC provides results in a collision 
rate of 1.7 GHz" ...kde jsi vzal 1.7 GHz? Tim asi myslis pocet 
jednotlivych srazek proton-proton... Nicmene to je asi zbytecne matouci 
informace, protoze u LHC se standardne udava to cislo 40 MHz, coz je 
frekvence srazek bunchu - frekvence eventu.
\item odkud mas informace o trigger systemu? Ten trifazovy L1/L2/EF, to byl 
trigger system v Runu 1. V Runu 2 uz mame jen dvoukrokovy - L1 a 
High-level trigger (HLT).
\green{Z roku 2008, povodny clanok. \url{https://cds.cern.ch/record/1129811?ln=en}, updatol som z \url{https://cds.cern.ch/record/2238679/files/ATL-DAQ-PROC-2016-039.pdf}}
\end{itemize}

\subsection{Chap 3}
\begin{itemize} 
\item \texttt{PartonTruthLabelID} info
\item Tab. 3.1 prekontroloval jsem cross-section a filter eff. a tyto udaje 
jsou v poradku. Ty pt ranges - odkud jsi je vzal? To jsem Ti posilal?
\green{Nie, nasiel som to v nejakom clanku. Preto som to chcel aj prekontrolovat.} \url{https://www.google.com/url?sa=t&rct=j&q=&esrc=s&source=web&cd=&cad=rja&uact=8&ved=2ahUKEwiJ-KGwp6T-AhW7gf0HHd5uDrwQFnoECAsQAQ&url=https%3A%2F%2Fera.library.ualberta.ca%2Fitems%2F64bff467-2b91-423d-87fe-77f15eb77472%2Fdownload%2F2d1e1293-0138-4033-b17a-875cd7fbd4b9&usg=AOvVaw1EMrRbyIuRdYgGlXCzaDoc}
\item p.36 "The seed of the cluster is a signal of strength at least 5sigma" 
...urcite to je 5 sigma? V Runu 1 to byvalo 4 sigma.
\green{Mas pravdu, v PFO clanku je 4.}
\item p.38 "Note that tracks are assigned only to charged with" ...charged 
particles, ne?
\item p.38 "The primary vertex is assigned an event selected by the 
trigger." ...tohle bych asi nezduraznoval, resp. vyhodil bych to. Ve 
vetsine pripadu to asi bude platit, ale na otagovani vertexu jako 
"primary" se pouziva jine pravidlo: ze ten vertex musi mit nejvyssi 
soucet druhych mocnin pricnych hybnosti asociovanych tracku
\item p.38 "The jet-vertex-fraction JVF is a fraction of the jet momentum" 
...of the jet transverse momentum
\item p.39 cuty na JVT a fJVT jsou oba (f)JVT > 0.5. Ale u fJVT je jeste cut 
na Timing (<10ns). V pripade potreby muzes kouknout sem: 
\url{https://twiki.cern.ch/twiki/bin/view/AtlasProtected/PileupJetRecommendations#Analysis_level_variable_calculat}
U obojiho jsem pouzil Tight working point.
\item p.39 "average number of interactions per bunch crossing denoted by µ." 
...average expected number...
\item p.40 "In case of cones overlapping, they are split if the overlapping 
energy is more than a given threshold (usually 0.5 of total energy)." 
...tady jsi asi chtel rict "they are merged", ne?
\item p.40 Cam/Aachen -> Cambridge/Aachen
\item p.41 pokud vim, tak zadny trimming jetu nepouzivame. Nakolik vim, tak 
ten se dela u "large-R" jetu, ktere se rekonstruuji za ucelem 
rekonstrukce hadronovych rozpadu W, Z a topu.
\end{itemize}

\subsection{Chap 4}
\begin{itemize} 
\item p.48 text o Adam algoritmu ...nemas v nem moc zmineno, co jsou ty 
promenne s, r, m a lambda z tabulky. Priznavam, ze ja jsem ten Adam 
algoritmus z tohoto popisu nepochopil. Pokud aspirujes na to, aby ctenar 
jisty feeling ziskal, pak by to chtelo jit lehoulince vic do detailu, 
obavam se.
\green{Pokus cislo dva...Ak stale nieje jasne, tak napis prosim trosku konkretnejsie nech viem v com je problem. To lambda popisujem presnejsie v sekcii Regularization}
\item Eq. 4.19 a 4.20 tam by asi melo byt $F_i$ a ve 4.21 by zase melo byt $T_i$ \green{To si praveze nemyslim, uz som to raz opravoval. Nadtymto som bol dlho zaseknuty ze ako je to definovane. Podla wiki tam ma byt opacna classa. tabulka vpravo \url{https://en.wikipedia.org/wiki/Sensitivity_and_specificity}}
\item Eq. 4.22 opravdu se to zapisuje timto zpusobem? Nemelo by se do toho 
zapisu zavest to, ze kazda slozka i $x^{l+1}_i$ se pocita pouze ze slozky 
i vektoru $Wx+b$? \green{To je implicitne myslene ak pozijes ReLU alebo sigmoid, ale softmax je definovany pre cely vektor. Ako tieto zapisy su v DL velmi vágne, ci to aplikujes na vektor alebo po zlozkach... 
Zvacsa si zavadzaju specialne symboly pre nasobenie $\odot$ ale ked mas funckiu tak je to zlozitejsie.}
Explicitne napisat, ze je to po zlozkach
\end{itemize}

\subsection{Chap 5}
\begin{itemize} 
\item "Previous work primarily used only one variable to tag the jet" -> 
Previous ATLAS work...
\item Fig. 5.1 v tom policku Cut Ti asi chybi -999
\item p.71 ten cut JVT>0.8, to jsem Ti asi poradil ja, vid? Ja ho pouzivam v 
tech fake tau studies, kde jsem si (driv) myslel, ze ma vyznam pozadovat 
vysoke JVT. Nicmene pro Tvoji praci by asi bylo lepsi nechat JVT cut 
takovy, jaky pouziva to Tight kriterium, tj. 0.5. No, ted uz to asi 
menit nebudeme, nejspis by to stejne nezmenilo vubec nic, protoze jetu s 
$JVT \in (0.5, 0.8)$ stejne moc neni.
\green{Hej, ked sme sa snazili vylepsit low pt performance.}
\item p.71 proc pozadujes pocet PFO vetsi nez 5? V textu to staci napsat 
tak, jak jsi to popsal, ale az se uvidime, tak si rad poslechnu vic.
\green{To som si vycucal z prstu :D , asi som sa chcel zbavit nejakych nizko energetickych jetov. Ale ked teraz nadtym premyslam je asi BS, lebo tie by sa skorej delili na viac partonov.}
\item p.71 "We do not want to train on pileup jets" ...tady se slusi rict, 
proc to nechceme. U pileup jetu se totiz v ATLAS MC neuklada truth 
informace. To je kvuli vyznamnemu usetreni na velikosti dat.
\item p.72 "Combining JZ slices with equal probability ensures the model 
works on the full spectrum of jets" ...to je rozhodne pravda! Dalsi 
motivaci pak je, aby se nas tagger nerozhodoval podle pt, jestli jet 
otaguje, jako kvarkovy, nebo gluonovy. Samozrejme cross-talk pt a 
ostatnich promennych je umoznen, ale je dulezite, aby se ten tagger 
nerozhodoval prvoplanove typu: "Ha, ten jet ma nizke pt, a tak ho 
otaguji jako gluonovy." Ted koukam, ze to tam na p.72 uplne dole mas, 
ale stalo by asi za to rict tuto informaci hned na zacatku odstavce.
\item Fig. 5.4 caption: toto jsou taky spektra *before* JZ cuts are applied, ne?
\item p.74 "variables that are variations of the 4-momenta of the PFOs" 
...tohle je podle me velmi nejasne az matouci. Co mas na mysli pod tim 
pojmem "variations"? Ja bych se snazil tomuto vyrazu maximalne vyhnout - 
v casticove fyzice se totiz delaji nejruznejsi "variations" ctyrhybnosti 
kvuli odhadu systematickych chyb.
\item p.75 "forming an input shape of N x 6," ...ale v tabulce 5.1 mas 
vyjmenovanych 8 promennych, tak kam zmizely ty dve a ktere to jsou?
\item p.75 "We want the tagger to be rotationally invariant, so we only use 
angles information wrt. the jet axis" ...tohle je spravny krok! Pak tu 
je ale problem, ze jsme nezajistili, aby eta-spektra q- a g-jetu byla 
stejna. Tvuj tagger se tedy urcite rozhoduje na zaklade eta, jaky tag mu 
da. Urcite bych to v bakalarce vubec nezminoval, ale pro tu ATLAS Public 
Notu to bude dulezite. Dokonce bys asi mel popremyslet, jestli nelze ta 
stejnost eta spekter q- a g-jetu nejak zajistit...
\green{Tu bude problem s korelaciou medzi pt a eta}
\item p.75 "assigned zero masses since they are reconstructed as photons" 
...jako fotony? Co to vubec znamena "byt zrekonstruovany jako foton"? 
Spis je to asi tak, ze kdyz nevime, jake castici ten PFO odpovida, tak 
ucinime nejjednodussi predpoklad - ze to proste je hodne relativisticka 
castice.
\item p.75 dole "of N x N x 6" ...tady by asi melo byt 4, ne? Soude podle 
tabulky 5.2...
\item p.75/76 jaky smysl ma zavadet trojhybnost? Pouzivas ji jen v tab. 5.2 
pro vypocet invariantni hmotnosti. Myslim si, ze by stacilo rict, ze m 
je inv. hmotnost toho paru a spocita se jako $(p_a + p_b)^2$, kde $p_i$ jsou 
ctyrhybnosti.
\green{Konvencia z orig clanku.}
\item p.76 v textu mas promennou $C_1^{\beta=1}$, ale v tab. 5.3 mas beta=0.2
\end{itemize}

\subsection{Conclusion}
\begin{itemize} 
\item p.91 urcite musis rict, ze jsi pouzival ATLAS simulace pro Run 2. 
Predevsim slovo "ATLAS" musi v Conclusions byt! \green{Ani som si nevsimol ze som to tam zabudol :D}
\item p.91 "which could be done relatively quickly by adding data 
augmentation steps" ...urcite vynech "which could be done relatively 
quickly"! Spis bych to nahradil necim jako "which might be achieved by 
adding data augmentation steps".
\item p.92 "we would test the performance on real data and calibrate the 
real data" ...kalibruje se temer vzdy jen MC. V nasem pouziti znamena 
pojem "kalibrace" prevazit MC tak, aby ucinnost selekce pomoci toho 
taggeru byla v MC stejna, jako je v datech.
\end{itemize}





\section{17.4.2023}


\subsection{Chap 1}
\begin{itemize}
    \item "Higgs boson is a scalar boson, giving invariant mass to other particles." ...vynech "invariant". Pod tim pojmem "mass" se v tomto kontextu v nasi hantyrce vzdy rozumi prave jedna vec. Jeste bys to mohl rozsirit a rict, ze Higgs boson je pozorovatelny artefakt Higgsova mechanismu (pricemz Higgsuv mechanismus je ten zpusob, jak do SM zavest hmoty castic)
    \item "In QED, the coupling constant g goes to zero as the
    distance between the particles goes to infinity." ...tohle teda taky nevim, jestli je pravda! Spis ne!! Jisty si jsem jen jednim: cim jsou od sebe nabite castice dal, tim je elmag. sila mezi nimi mensi. To je ale dane tou vzdalenosti a ne nutne tim, ze by jeste navic klesala i vazbova konstanta. Ne, pockej, ta vazbova konstanta asi trochu klesa, ale urcite nejde do nuly. Kazdopadne: rozlisuj peclive mezi poklesem/rustem vazbove konstanty a poklesem/rustem sily, kterou na sebe castice pusobi
\end{itemize}

\subsection{Chap 2}
\begin{itemize}
    \item "ATLAS MC samples do not come with truth information about pileup to reduce
    the amount of used disk space," ...about pileup jets
\end{itemize}