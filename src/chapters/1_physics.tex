\chapter{Physics of Quarks and Gluons}

\section{The Standard Model}
\label{sec:sm}
The best widely adopted and tested description of the fundamental particles and their interactions is the \SM \cite{pdg}. 
\begin{figure}[htb]
    \centering
    \includegraphics[width=0.6\linewidth]{/Users/samueljankovych/Documents/MFF/bakalarka/thesis/src/img/sm.jpg}
    \caption[The Standard Model of particle physics.]{The Standard Model of particle physics. \footnotemark}
    \label{fig:sm}
\end{figure}
\footnotetext{\url{https://en.wikipedia.org/wiki/File:Standard_Model_of_Elementary(P)articles.svg}}
\emph{Fermions} are particles that makeup all the matter and \emph{bosons} are force-mediating particles.
All known elementary particles are displayed in \cref{fig:sm}.

Fermions are divided into two categories, \emph{leptons} and \emph{quarks}. 
Leptons are further divided into two groups:
\begin{enumerate}
    \item  
    \begin{itemize}
        \item \emph{electron}, a particle that is present in all matter around us,
        \item \emph{muon}, a heavier cousin of electron naturally produced in the atmosphere,
        \item \emph{tau lepton}, the heaviest lepton.
    \end{itemize}
    These three leptons can interact via the \emph{Electromagnetic Force} (they have a charge), mediated by \emph{photons}, or via the \emph{Weak force}, mediated by \emph{W and Z boson}. 
    \item \emph{Neutrinos}, each belonging to either electron, muon, or tau lepton.  
    Neutrinos can interact only weakly by exchanging W and Z bosons.
\end{enumerate}
Quarks are also divided into two groups:
\begin{multicols}{2}
\begin{enumerate}
    \item  
    With electric charge $\frac{2}{3}$ \footnote{Measured in units of elementary charge $e=1.60217663\cdot10^{-19}$ C.},
    \begin{itemize}
        \item \emph{up}, u quark,
        \item \emph{charm}, c quark,
        \item \emph{top}, t quark.
    \end{itemize}
    \item  
    With electric charge $-\frac{1}{3}$.
    \begin{itemize}
        \item \emph{down}, d quark,
        \item \emph{strange}, s quark,
        \item \emph{bottom}, b quark.
    \end{itemize}
\end{enumerate}
\end{multicols}
All quarks can interact weakly and electromagnetically via the \emph{Strong Force}, mediated by \emph{gluons}.
In nature, practically only the u-quark and d-quark occur. 
Concretely, they are the building blocks of protons and neutrons.

Fermions can be also split another way, into three families. 
The mass of corresponding particles gives the order of families.
The lightest is built up by a u-quark, a d-quark, an electron, and an electron neutrino.
Practically all matter around us is made up of particles from this family.
It should be noted that neutrino masses are not precisely known.
Due to a measured process \cite{sadbury} called neutrino oscillations \cite{pdg}, they cannot be massless.
Since the first discovery, estimates on the upper bound of their masses were made \cite{pdg}.

The bosonic particles are a photon, gluon, W, and Z boson, and \emph{Higgs boson}. They can be split up into two groups:
\begin{enumerate}
    \item \emph{Vector Bosons}: photon, gluon, W and Z boson,
    \item \emph{Gauge Bosons}: Higgs boson.
\end{enumerate}
Vector bosons ('vector' comes from their mathematical description) are force carriers:
\begin{itemize}
    \item \emph{Electromagnetic Force} - photon,
    \item \emph{Strong Force} - gluon,
    \item \emph{Weak Force} - Z and W boson.
\end{itemize}
The properties of the corresponding force-carrying boson give the nature of each force.
The corresponding charge of interacting particles gives the interaction strength of each force.

Particles that have electric charges can exchange photons and interact electromagnetically.
The infinite range of the Electromagnetic force is due to the massless photons.
They also have no charge, which makes them unable to scatter on other photons.~\footnote{This has some caveats at higher energies. Classically it is forbidden due to the linearity of Maxwell's equations, but at higher orders of perturbation QFT, photons can scatter on another photon. However, the probability of such an event is very low.}

Gluons are also massless, making the Strong force also infinite.
Quarks are the only fermion carriers of the strong charge (also called \emph{color charge}). 
Since gluons have two strong chargers, they can interact with other gluons and exchange color charges when interacting with quarks or gluons.
This differs from Electromagnetic Interaction, where particles' charge does not change. 
Even though the Strong interaction is theoretically infinite, we do not see free gluons or quarks due to a process called \emph{hadronization} (detailed description in ????). 
They only occur in a bound state in nature, effectively making the color charge invisible at a larger scale (it becomes visible at~the~scale~of approximately 1 fm)

All fermions can interact weakly, but since W and Z bosons have mass (which is significant compared to fermions), the range of interaction is short.
Weak processes occur, for example, in $\beta$-decay.

Last but not least is the Higgs boson, a scalar boson.
The strength of the interaction with the Higgs boson (more precisely: with the scalar field, Higgs field) is given by the mass of the interacting particle.
In other words, particles' interaction with the Higgs field gives them mass.
Gluons and photons are the only particles from \SM non-interacting with the Higgs boson.



\section{Quantum Field Theory}
\label{sec:qft}
\QFT is a fundamental framework in modern theoretical physics that describes the behavior of particles and fields at the quantum level.
Its first development was done by studying the electromagnetic interaction of particles and photons by Paul Dirac \cite{dirac}.
Massive progress in \QED was made by Richard Feynman \cite{feynman}, who also introduced the \emph{Feynman diagrams}, a graphical representation of interactions.
We will discuss Feynman diagrams in more detail in \cref{sec:feyman}.

The starting point of QFT is that particles are not fundamental objects but instead are excitations of underlying quantum fields that permeate all of space and time. 
This field must satisfy the equations of motion of the corresponding field theory derived from the Lagrangian density of the theory.

\subsection{Free Particles}
\label{sec:free_particles}
The Lagrangian of non-interacting \spinhalf particle also called the \emph{Dirac Lagrangian}, is given by
\begin{equation}
    \label{eq:dirac_lag}
    \mathcal{L}_{\text{Dirac}} = \bar{\psi}(i\slashed{\partial} - m) \psi,
\end{equation}
where we abbreviate $\gamma^\mu \partial_\mu \equiv \slashed{\partial}$, $\gamma^\mu$ are the Dirac matrices, and $\partial_\mu$ is the covariant derivative.
$\psi(x)$ is a bispinor fermionic field of a spin $\frac{1}{2}$ particle. 
It has four components, which can be split into two spinors, each corresponding to a different chirality
\begin{equation}
    \psi(x) = \left( \begin{array}{c} \psi_L(x) \\ \psi_R(x) \end{array} \right).
\end{equation}
Going further we will always abbreviate $\gamma^\mu V_\mu \equiv \slashed{V}$, where $V$ is a 1-form or a vector with lowered indices $V_\mu = \eta_{\mu \nu} V^\nu$ with the Minkowski metric $\eta_{\mu \nu} = \text{diag}(1,-1,-1,-1)$. 
We also automatically sum over repeated indices, utilizing the Einstein summation convention.
The Dirac equation is an equation of motion derived from the Lagrangian \cref{eq:dirac_lag}:
\begin{equation}
    \label{eq:dirac}
    \left( i \gamma^\mu \partial_\mu - m \right) \psi(x) = 0.
\end{equation}
It can be solved analytically.

The solutions to this equation describe how fermion particles propagate in space and time without interaction. 

Conversely, bosons are spin-1 particles, described by the vector field $A_\mu(x)$.
The Lagrangian of a non-interacting spin-1 particle is given by a \emph{Proca Lagrangian}
\begin{equation}
    \label{eq:proca_lag}
    \mathcal{L}_{\text{Proca}} = - \frac{1}{4} F_{\mu\nu} F^{\mu\nu} + \frac{1}{2} m^2 A_\mu A^\mu,
\end{equation}
where $F_{\mu\nu} = \partial_\mu A_\nu - \partial_\nu A_\mu$ is the field strength tensor.
Note that this Lagrangian describes massive bosons. 
In the case of massless bosons, the second term is omitted.
The equation of motion for the vector field, the Klein-Gordon equation, follows from the Lagrangian \cref{eq:proca_lag}
\begin{equation}
    \label{eq:klein_gordon}
    \left( \Box + m^2 \right) A_\mu(x) = 0.
\end{equation}
The solutions to this equation can be also found analytically.
In the case of massless bosons, the equation reduces to 
\begin{equation}
    \label{eq:wave_eq}
    \Box A_\mu(x) = 0,
\end{equation}
which is a simple wave equation.
We can rewrite it using the field strength tensor as
\begin{equation}
    \Box A_\mu(x) = \partial_\nu \partial^\nu A_\mu(x) = \partial_\nu \partial_\nu A_\mu(x) - \partial_\nu \partial_\mu A_\mu(x) =  \partial_\nu F_{\nu\mu}(x) =  0,
\end{equation}
which are Maxwell's equations. 
This means that massless non-interacting bosons propagate the same way as electromagnetic waves.
We have utilized the Lorentz gauge $\partial_\mu A_\mu(x) = 0$.


\subsection{Interacting Particles}
\label{sec:interacting_particles}
We introduce the interaction between fields by invoking the \emph{local gauge invariance} of the Lagrangian. 

Each type of interaction (strong, electromagnetic, weak,...) corresponds to a different $SU(N)$ \emph{Lie group} \cite{fecko}.
Elements of $SU(N)$ are special unitary $N$x$N$ matrices \footnote{Matrix U is special if det$U=1$ and unitary if $UU^{\dagger} = 1$}. 
When operating on $N$-dimensional space of fields, they are represented as unitary operators in the form 
\begin{equation}
    \label{eq:lie_element}
    U = \eu^{i g \sum_{a} \alpha^a T^a},
\end{equation}
where $T^a$ are generators of the group, and $\alpha^a$ and $g$  are the parameters of the transformation.
For example in the case of a rotating spinor in classical quantum mechanics, the group is the rotation group SU(2), and the generators are the Pauli matrices $T^a \equiv \frac12 \hat{\sigma}_a$ ($a = x,y,z$).
$g$ would be the angle of rotation, and $\alpha^a \equiv \vec{n}$ is the axis of rotation.
More generally $T^a$ can be represented as $N$x$N$ matrices, where $N$ is the field space dimension.
Going further, we will suppress writing the summation over $a$ (or any other index) explicitly, unless it is necessary for clarity.
The generators satisfy the commutation relations
\begin{equation}
    \label{eq:commutation}
    [T^a, T^b] = i f^{abc} T^c,
\end{equation}
where $f^{abc}$ are the structure constants of the Lie group.

A Lagrangian describes an interacting physical system if it is invariant under the local gauge transformation
\begin{equation}
    \label{eq:}
    \psi_j(x)' = U_{ij}(x)\psi(x)_j = \eu^{i g \alpha^a(x) T^a_{ij}} \psi(x)_j,
\end{equation}
of the spinor quantum fields $\psi(x)_i$, and under the infinitesimal transformation of the vector fields $A^a_\mu(x)$
\begin{equation}
    \label{eq:gauge_inv_bosons}
    A^a_\mu(x)' \rightarrow A^a_\mu(x) + \frac{1}{g} \partial_\mu \alpha^a(x) - f^{abc} \alpha^b(x) A^c_\mu(x) + \mathcal{O}((f^{abc})^2),
\end{equation}
where $U(x)$ is an element of the corresponding Lie group for every point in spacetime $x$.
$g$ is the coupling constant of the interaction, and $\alpha^a(x)$ are the parameters of the transformation.


For example, the electromagnetic interaction corresponds to $U(1)$ group (Lie group isomorphic to complex numbers), which means that the Lagrangian must be invariant under the transformations
\begin{equation}
    \label{eq:gauge_inv_QED}
    \psi(x) = \eu^{-i e \alpha(x)}\psi(x) \quad, \quad A_\mu(x) = A_\mu(x) + \frac{1}{e} \partial_\mu \alpha(x).
\end{equation}
where the a generator of the $U(1)$ group is just a constant $T^a = 1$, so the structure constants are zero $f^{abc} = 0$.

In the general case of the non-Abelian $SU(N)$ group having $N$ different fermions with the free Lagrangian
\begin{equation}
    \label{eq:Dirac_N}
    \mathcal{L}_{\text{Dirac}} = \sum_{i=1}^N \bar{\psi}_i (i\slashed{\partial} - m) \psi_i,
\end{equation}
the local gauge invariant Lagrangian is given by \cite{schartz}
\begin{equation}
    \label{eq:lag_suN}
    \mathcal{L} = - \frac{1}{4}  G^a_{\mu\nu} G^a_{\mu\nu} + \frac{1}{2} m^2 A^a_\mu A^a_\mu, + \sum_{i,j=1}^N \bar{\psi}_i (i\delta_{ij} \slashed{\partial}  +g \slashed{A}^a T^a_{ij} - m \delta_{ij}) \psi_j.
\end{equation}
where the field strength tensor is modified, as a result of the non-commutativity of the generators, to
\begin{equation}
    \label{eq:field_strength}
    G^a_{\mu\nu} = F^a_{\mu\nu} + g f^{abc} A^b_\mu A^c_\nu = \partial_\mu A^a_\nu - \partial_\nu A^a_\mu + g f^{abc} A^b_\mu A^c_\nu,
\end{equation}
such that it preserves the local gauge invariance.
This has a huge impact on the dynamics of the system because the last term corresponds to boson-boson interaction.
We shall see in \cref{sec:QCD} the consequences of this interaction.

To go back to the example of $U(1)$, the corresponding Lagrangian has the form \footnote{$G^a_{\mu\nu} \equiv F^a_{\mu\nu}$, because $f^{abc} = 0$}
\begin{equation}
    \label{eq:qed_lag}
    \mathcal{L} = - \frac{1}{4}  F_{\mu\nu} F_{\mu\nu} + \bar{\psi}(i\slashed{\partial} - m) \psi - e \bar{\psi} \gamma^\mu \partial_\mu \psi A_\mu.
\end{equation}
The equations of motion of the Lagrangian (\ref{eq:qed_lag}) completely determine the dynamics of the system.

\subsection{Cross Section}
\label{sec:cross_section}
In particle physics, we are not interested in calculating the precise state $psi(x)$ as a solution to the equations of motion of the Lagrangian (\ref{eq:qed_lag}), but rather in the probability of two particles interacting. 
The physical quantity describing this probability is the cross section $\sigma$.
With the probabilistic nature of quantum mechanics, we need to define this quantity based on a larger number of interactions.
Suppose we have a flux of incoming particles that are scattered off of target particles. 
Then the probability that a given particle will scatter off of a target particle is given as 
\begin{equation}
    \label{eq:cross_sec}
    \sigma = \frac{\text{number of scattered particles}}{\text{incoming flux of particles}},
\end{equation}
where the flux has units m$^{-2}$ so the cross section has units m$^{2}$.

If we want to specify the infinitesimal probability of particles being scattered to a space angle d$\Omega$, we can define it as
\begin{equation}
    \label{eq:diff_cross_sec}
    d\sigma = \frac{\text{number of scattered particles to an space angle } d\Omega}{\text{incoming flux of particles}}.
\end{equation}
Or if we want the cross section in terms of the momentum of the scattered particles d$p$, we can define it as 
\begin{equation}
    \label{eq:diff_cross_sec_p}
    d\sigma = \frac{\text{number of scattered particles with outgoing momentum } dp}{\text{incoming flux of particles}}.
\end{equation}

Quantum mechanically we expect the cross section to be given by the probability of the system being in a final state $|f\rangle$ given that was in the initial state $|i\rangle$.
We can express this as 
\begin{equation}
    \label{eq:cross_sec_qm}
    \sigma \sim |\langle f | \eu^{-i \hat{H} (t_{out} - t_{in})} | i \rangle |^2,
\end{equation}
where $\hat{H}$ is the Hamiltonian of the system, and $t_{in}$ and $t_{out}$ are the initial and final times of the system.

To be more precise, for two interacting particles with energies $E_{1,{\text{in}}}$, $E_{2,{\text{in}}}$ and velocities $\vec{v}_{1,{\text{in}}}$, $\vec{v}_{2,{\text{in}}}$, and $N$ particles in the final state with energies $E_{j,{\text{out}}}$, $j=1,2,...,N$ , we can write \cite{schartz}
\begin{equation}
    \label{eq:cross_sec_qft}
    d \sigma = \frac{|\mathcal{M}|^2}{4 E_{1,{\text{in}}} E_{2,{\text{in}}} |\vec{v}_{1,{\text{in}}} - \vec{v}_{2,{\text{in}}}|} (2\pi)^4 \delta^4(p^\mu_{\text{in}} - p^\mu_{\text{out}}) \prod_{j=1}^N \frac{d^3 p_j}{(2\pi)^3 2E_{j, {\text{out}}}}, 
\end{equation}
where $p^\mu_{\text{in}}$ is the 4-momentum of the incoming particles and $p^\mu_{\text{out}}$ is the 4-momentum of the outgoing particles, the $\delta^4(p^\mu_{\text{in}} - p^\mu_{\text{out}})$ function is the 4-dimensional Dirac delta function enforcing 4-momentum conservation, and the $|\mathcal{M}|^2$ is the matrix element of the \emph{scattering amplitude} $\mathcal{M}$.
The matrix element $\langle f |\mathcal{M} | i \rangle$ (where $|\mathcal{M}|^2 \equiv |\langle f |\mathcal{M} | i \rangle |^2$) is given by (assuming $ | f \rangle \neq | i \rangle$)
\begin{equation}
    \label{eq:M_matrix}
    \langle f | S | i \rangle = i (2\pi)^4 \delta^4(p^\mu_{\text{in}} - p^\mu_{\text{out}})  \langle f | \mathcal{M} | i \rangle,
\end{equation}
where $S$ is the $S$-matrix of the scattering process.
We can calculate the $S$-matrix utilizing the Dyson series expansion \cite{schartz}
\begin{equation}
    \label{eq:s_dyson}
    \langle f | S | i \rangle =\left\langle f \left|\sum_{n=0}^{\infty} \frac{(-i)^n}{n !} \int d x_1^4 \cdots \int d x_n^4 T\left[\mathcal{H}\left(t_1\right) \cdots \mathcal{H}\left(t_n\right)\right]\right| i \right\rangle,
\end{equation}
where $T\left[\mathcal{H}\left(t_1\right) \cdots \mathcal{H}\left(t_n\right)\right]$ is the time-ordered product of the Hamiltonian of the system, and $\mathcal{H}$ is the Hamiltonian density of the system.
Equation (\ref{eq:cross_sec_qm}) is a special case of equation (\ref{eq:s_dyson}) where the Hamiltonian is time-independent.


\subsection{Feynman Diagrams}
\label{sec:feyman}
Calculating the expansion (\ref{eq:s_dyson}) is practically impossible to do analytically, so Feynman came up with a graphical method to calculate individual terms.
The diagrams consist of:
\begin{itemize}
    \item \textbf{External lines} represent the incoming and outgoing particles, which are \emph{real} particles,
    \item \textbf{Internal lines} represent the intermediate particles, which are \emph{virtual} particles,
    \item \textbf{Vertices} are the points of interaction.    
\end{itemize}
Virtual particles are not real particles but a mathematical construct that allows us to visualize the scattering process.
Each of these parts contributes a multiplicative factor to the amplitude $-i\mathcal{M}$.
The \emph{order of the diagram} is the number of interaction vertices.
This is also the order of the term in the Dyson series expansion.

\begin{figure}[htb]
\begin{tikzpicture}
    \begin{feynman}
        \vertex (a);
        \vertex [below =of a](c);
        \vertex [above left=of a] (i1) {\(\mu^{-}, p_1\)};
        \vertex [above right=of a] (f1) {\(\mu^{-}, p_3\)};
        \vertex [below left=of c] (i2) {\(e^{-}, p_2\)};
        \vertex [below right=of c] (f2) {\(e^{-}, p_4\)};
        \diagram* {
            (i1) -- [fermion] (a) -- [fermion] (f1),
            (a) -- [photon, edge label=\(\gamma\), momentum'=\(p_\gamma\)] (c),
            (i2) -- [fermion] (c) -- [fermion] (f2),
        };
    \end{feynman}
\end{tikzpicture}
\caption{Second-order Feynman diagram of electron-muon scattering.}
\label{fig:electron_muon}
\end{figure}
To give an example we consider electron-muon scattering.
The second-order Feynman diagram for this process is shown in \cref{fig:electron_muon}.
This is the lowest-order diagram that can contribute to the cross section.

The time flows from left to right, so we have an incoming electron and muon, they exchange a virtual photon, momentum, and scatter.
The incoming particles, with momenta $p_1$, $p_2$, contribute factors $u(p_1)$, $u(p_2)$, which are solutions to the free Dirac equation (\ref{eq:dirac}) in momentum space. 
The outgoing particles, with momenta $p_3$, $p_4$, contribute factors $\bar{u}(p_3)$, $\bar{u}(p_4)$, which are complex conjugates of the free Dirac equation solutions.
The vertex is where interaction happens, and corresponding term $ie\gamma^\mu$ is given by the interaction Lagrangian $\mathcal{L}_{\text{int}} = - e \bar{\psi} \gamma^\mu \partial_\mu \psi A_\mu$.
Virtual photon, with momentum $p_{\gamma}$, is an internal line, its contribution is $\frac{-i\eta_{\mu \nu}}{p_{\gamma}^2}$.
We can now write the amplitude as 
\begin{equation}
    \label{eq:e_mu_amp}
    -i\mathcal{M} = \bar{u}(p_3) (ie\gamma^\mu) u(p_1) \frac{-i\eta_{\mu \nu}}{p_{\gamma}^2} \bar{u}(p_4) (ie\gamma^\nu) u(p_2)
\end{equation}
Utilizing the conservation of momentum, we can write $p_{\gamma} = p_1 - p_3$, we can write
\begin{equation}
    \label{eq:e_mu_amp_2}
    \mathcal{M} = -e^2 \bar{u}(p_3) \gamma^\mu u(p_1) \frac{1}{(p_1 - p_3)^2} \bar{u}(p_4) \gamma_\mu u(p_2).
\end{equation}
After multiplying by the complex conjugate, summing over all the possible spins, and neglecting masses of electron and muon (we assume relativistic scattering), we can write the matrix element as
\begin{equation}
    |\mathcal{M}|^2 = 2e^4\frac{(p_1 + p_2)^2 + (p_1 - p_4)^2}{(p_1 - p_3)^2}.
\end{equation}
We can now evaluate the cross section (\ref{eq:cross_sec_qft}) in the centre of mass frame as
\begin{equation}
    \label{eq:cross_sec_qft_2}
    \frac{d\sigma}{d\Omega} = \frac{\alpha^2}{4(p_1 + p_2)^2} (1+\cos^2{\theta}),
\end{equation}
where $\theta$ is the scattering angle.
Integrating over the solid angle, we get the total cross section
\begin{equation}
    \sigma =  \frac{4\pi\alpha^2}{3(p_1 + p_2)^2}.
\end{equation}

\begin{figure}[htb]
    \begin{subfigure}[t]{0.4\textwidth}
        \begin{tikzpicture}
            \begin{feynman}
                \vertex (a);
                \vertex [below =of a](c);
                \vertex [right =of a](b);
                \vertex [right =of c](d);
                \vertex [above left=of a] (i1) {\(\mu^{-}\)};
                \vertex [above right=of b] (f1) {\(\mu^{-}\)};
                \vertex [below left=of c] (i2) {\(e^{-}\)};
                \vertex [below right=of d] (f2) {\(e^{-}\)};
                \diagram* {
                    (i1) -- [fermion] (a) -- [fermion] (b) -- [fermion] (f1),
                    (a) -- [photon] (c),
                    (b) -- [photon] (d),
                    (i2) -- [fermion] (c) -- [fermion] (d) -- [fermion] (f2),
                };
            \end{feynman}
        \end{tikzpicture}
        \caption{}
        \label{fig:e_mu_higher_a}
    \end{subfigure}
    \begin{subfigure}[t]{0.4\textwidth}
        \begin{tikzpicture}
            \begin{feynman}
                \vertex (a);
                \vertex [below =of a](c);
                \vertex [right =of a](b);
                \vertex [right =of c](d);
                \vertex [above left=of a] (i1) {\(\mu^{-}\)};
                \vertex [above right=of b] (f1) {\(\mu^{-}\)};
                \vertex [below left=of c] (i2) {\(e^{-}\)};
                \vertex [below right=of d] (f2) {\(e^{-}\)};
                \diagram* {
                    (i1) -- [fermion] (a) -- [fermion] (b) -- [fermion] (f1),
                    (a) -- [photon] (d),
                    (b) -- [photon] (c),
                    (i2) -- [fermion] (c) -- [fermion] (d) -- [fermion] (f2),
                };
            \end{feynman}
        \end{tikzpicture}
        \caption{}
        \label{fig:e_mu_higher_b}
    \end{subfigure}
    \caption{Fourth order Feynman diagram for electron-muon scattering.}
    \label{fige_mu_higher}
\end{figure}


This calculation was an example of the lowest-order Feynman diagram.
In general, many more diagrams contribute to the amplitude, \cref{fig:electron_muon} is a fourth-order diagram.

In Apendix ???? we give the \emph{Feynman rules} for the general \QED Feynman diagram \cite{intro_to_part}.
Other interactions have slightly different Feynman rules, but the general idea is the same.


\section{Quantum Chromodynamics}
\label{sec:QCD}
So far we have introduced \QED interactions as examples. 
In this section, we will introduce \QCD, which is the theory of strong interactions.
The underlying Lie group of \QCD is  $SU(3)$.
Utilizing the general theory introduced in \cref{sec:interacting_particles}, we can derive the following. 
$N=3$, which means the underlying field space is three-dimensional.
Physically meaning we have \textbf{3 quark fields} for each quark in the \SM.
They correspond to 3 different color states $\psi_i(x)$, $i=1,2,3$.
$SU(3)$ has 8 generators $T^a$, which means we have \textbf{8 gluon fields} $A^a_\mu(x)$, $a=1,\dots, 8$.
The full Lagrangian of \QCD is \cite{qcd} (we also suppress the summation over quark fields $i,j$)
\begin{align}
\begin{split}
    \label{eq:qcd_lag}
    \mathcal{L}_{\text{QCD}}  &= - \frac{1}{4}  G^a_{\mu\nu} G^a_{\mu\nu}  + \bar{\psi}_i (i \slashed{\partial} - m ) \psi_i + g \bar{\psi}_i \slashed{A}^a T^a_{ij} \psi_j \\
     &= - \frac{1}{4}  F^a_{\mu\nu} F^a_{\mu\nu}  + \bar{\psi}_i (i \slashed{\partial} - m ) \psi_i + g \bar{\psi}_i \slashed{A}^a T^a_{ij} \psi_j + \\    
     &+ g f^{a b c}\left[\left(\partial_\mu A_\nu^a-\partial_\nu A_\mu^a\right) A_b^\mu A_c^\nu+\left(\partial_\mu A^{a\nu}-\partial_\nu A^{a \mu}\right) A_{b \mu} A_{c \nu}\right] + \\
     &+ g^2 f^{a b c} f^{a d e} A_b^\mu A_c^\nu A_{d \mu} A_{e \nu},
\end{split}
\end{align}
where we have expanded $G^a_{\mu\nu}$ as in (\ref{eq:field_strength}). 
We can now see that the two last terms correspond to three and four-gluon coupling.

\subsection{Infrared Divergence}
\label{sec:IR_div}
The perturbation expansion (\ref{eq:s_dyson}) is done in the constant of interaction $g$. 
However, the assumption of $g$ being a constant is not valid.
It changes with the energy of the particles, hence the distance between them. 
In \QED we have the \emph{asymptotic freedom}, which means the coupling constant $g$ goes to zero as the distance between the particles goes to infinity.
When we force particles to be closer to each other, by smashing them together with high enough energy, the coupling constant grows.
This is not the case for \QCD, where the coupling constant grows with the distance between the quarks.
The reason for this is that in \QCD there are three quark color fields, whereas in \QED there is only one.
The structure given by the Lie group $SU(3)$ dramatically changes the behavior of the coupling constant.
Another way to formulate this is that the probability of a quark emitting a gluon as it losses energy goes to infinity.
This is called \textbf{infrared divergence}.


\subsection{Feynamn Diagrams of QCD}
In Appendix ????, we will give the full list of Feynman rules for \QCD.
However, we will introduce some of the most important Feynman diagrams here.

% Lowest order diagrams for quark-quark scattering are shown in \cref{fig:qcd_qq}.
% \begin{figure}[htb]
%     \begin{subfigure}[t]{0.4\textwidth}
%         \begin{tikzpicture}
%             \begin{feynman}
%                 \vertex (a);
%                 \vertex [below =of a](c);
%                 \vertex [right =of a](b);
%                 \vertex [right =of c](d);
%                 \vertex [above left=of a] (i1) {\(\mu^{-}\)};
%                 \vertex [above right=of b] (f1) {\(\mu^{-}\)};
%                 \vertex [below left=of c] (i2) {\(e^{-}\)};
%                 \vertex [below right=of d] (f2) {\(e^{-}\)};
%                 \diagram* {
%                     (i1) -- [fermion] (a) -- [fermion] (b) -- [fermion] (f1),
%                     (a) -- [photon] (c),
%                     (b) -- [photon] (d),
%                     (i2) -- [fermion] (c) -- [fermion] (d) -- [fermion] (f2),
%                 };
%             \end{feynman}
%         \end{tikzpicture}
%         \caption{}
%         \label{fig:e_mu_higher_a}
%     \end{subfigure}
%     \begin{subfigure}[t]{0.4\textwidth}
%         \begin{tikzpicture}
%             \begin{feynman}
%                 \vertex (a);
%                 \vertex [below =of a](c);
%                 \vertex [right =of a](b);
%                 \vertex [right =of c](d);
%                 \vertex [above left=of a] (i1) {\(\mu^{-}\)};
%                 \vertex [above right=of b] (f1) {\(\mu^{-}\)};
%                 \vertex [below left=of c] (i2) {\(e^{-}\)};
%                 \vertex [below right=of d] (f2) {\(e^{-}\)};
%                 \diagram* {
%                     (i1) -- [fermion] (a) -- [fermion] (b) -- [fermion] (f1),
%                     (a) -- [photon] (d),
%                     (b) -- [photon] (c),
%                     (i2) -- [fermion] (c) -- [fermion] (d) -- [fermion] (f2),
%                 };
%             \end{feynman}
%         \end{tikzpicture}
%         \caption{}
%         \label{fig:e_mu_higher_b}
%     \end{subfigure}
%     \caption{Fourth order Feynman diagram for electron-muon scattering.}
%     \label{fige_mu_higher}
% \end{figure}