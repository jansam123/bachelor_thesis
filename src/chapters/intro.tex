
\chapwithtoc{Introduction}
\textcolor{red}{WILL BE UPDATED}

In order for physicists to understand the fundamental processes that govern the physical reality it is necessary to identify all the participants of these processes in experiments that probe them. 
\emph{ATLAS}~\cite{ATLAS} detector, operating on the \LHC~\cite{LHC} at \CERN~\cite{cern}, is the biggest experiment testing our knowledge of elementary interactions. 
The \LHC accelerates two proton beams in different directions, so they can collide in the ATLAS detector and create variety of particles.

Two fundamental particles that we are interested in are \emph{quarks} and \emph{gluons}~\cite{quarks}.
They do not occur free in the nature, but rather combine to create \emph{hadrons} in a process called \emph{hadronization}~\cite{hadronisation}.
If these particles are highly energetic, the hadronization process evolves and creates a shower of hadrons called \emph{jets}~\cite{jet}.
These jets have a vague experimental definition because it depends on the algorithm clustering hadrons together.
In our study we use the anti-$k_t$ algorithm~\cite{antikt} to define jets.
These jets are then detected in the ATLAS detector.

Our goal is to tell apart jets that are initiated by quarks and by gluons. 
Traces left by these particles in the detector are very similar.
This makes them hard to distinguish and non-trivial procedures are developed to make it possible.
In the past, hand-crafted variables (such as number of hadron tracks detected in the jet) describing the jets were used~\cite{ntrk_tag}.
Nowadays, modern \ml techniques are used to utilize more variables.
The most adopted and successful is \bdt algorithm~\cite{bdt}, which is still being developed~\cite{bdt_tag}.
It uses multiple \emph{jet variables} (variables that describe jet as a whole) as an input.

In this thesis we approach the problem using modern \dl techniques.
Firstly the \hgn architecture~\cite{highway} is developed, which still uses jet variables, but considerably more then \bdt. 
Afterwards we turn to more granular description of jet, using \emph{jet constituents}. 
These are particles measured by the detector and clustered into the jet. 
We use the jet constituents as an input for \trans-like architectures~\cite{att_is_all}.

Different architectures are explored.
Most of the techniques utilized are adopted from image classification tasks~\cite{deit3} and \nlp~\cite{bert}.
Recently this approach was also explored by the \CMS~\cite{cms} Collaboration, where they solve a slightly different problem~\cite{part}.
However, it can be directly applied to quark/gluon classification.
We implement their original architecture and slightly improve it.

These modern approaches vastly improve the older, established, \bdt and \hgn.   
They are also more scalable and flexible.






