\usepackage{abbrevs}
\usepackage{booktabs}
\usepackage{multicol}
\usepackage{slashed}
\usepackage[compat=1.0.0]{tikz-feynman}
\usepackage{subcaption}
\usepackage{xurl}
\usepackage{xcolor}
\usepackage{multirow}
\usepackage{placeins}

% \hypersetup{colorlinks=true, citecolor=green, urlcolor=blue, linkcolor=red}
% \hypersetup{hidelinks}
% \hypersetup{
%     bookmarks=true,         % show bookmarks bar?
%     unicode=false,          % non-Latin characters in Acrobat’s bookmarks
%     pdftoolbar=true,        % show Acrobat’s toolbar?
%     pdfmenubar=true,        % show Acrobat’s menu?
%     pdffitwindow=false,     % window fit to page when opened
%     pdfstartview={FitH},    % fits the width of the page to the window
%     pdftitle={My title},    % title
%     pdfauthor={Author},     % author
%     pdfsubject={Subject},   % subject of the document
%     pdfcreator={Creator},   % creator of the document
%     pdfproducer={Producer}, % producer of the document
%     pdfkeywords={keyword1, key2, key3}, % list of keywords
%     pdfnewwindow=true,      % links in new PDF window
%     colorlinks=false,       % false: boxed links; true: colored links
%     linkcolor=red,          % color of internal links (change box color with linkbordercolor)
%     citecolor=green,        % color of links to bibliography
%     filecolor=magenta,      % color of file links
%     urlcolor=cyan,           % color of external links
%     % urlbordercolor={1 1 1}  % color of border around links
% }


% Abbreviations

\newabbrev{\LHC}{\emph{Large Hadron Collider} (LHC)}[LHC]
\newabbrev{\CERN}{European Organization for Nuclear Research (CERN)}[CERN]
\newabbrev{\CMS}{\emph{Compact Muon Solenoid} (CMS)}[CMS]


\newabbrev{\bdt}{\emph{Boosted Decision Trees} (BDT)}[BDT]
\newabbrev{\dl}{\emph{Deep Learning}}[Deep Learning]
\newabbrev{\ml}{\emph{Machine Learning}}[Machine Learning]
\newabbrev{\hgn}{\emph{Highway Network}}[Highway Network]
\newabbrev{\trans}{\emph{Transformer}}[Transformer]
\newabbrev{\kNN}{\emph{k-Nearest Neighbors} (kNN)}[kNN]
\newabbrev{\mlp}{\emph{Multilayer Perceptron} (MLP)}[MLP]
\newabbrev{\depart}{\emph{Dynamicaly Enhanced Particle Transformer} (DeParT)}[DeParT]
\newabbrev{\ParT}{\emph{Particle Transformer} (ParT)}[ParT]

\newabbrev{\nlp}{Natural Language Processing (NLP)}[NLP]


\newabbrev{\IRC}{\emph{Infrared and Collinear safe} (IRC)}[IRC]
\newabbrev{\QCD}{\emph{Quantum Chromodynamics} (QCD)}[QCD]
\newabbrev{\QFT}{\emph{Quantum Field Theory} (QFT)}[QFT]
\newabbrev{\QED}{\emph{Quantum Electrodynamics} (QED)}[QED]
\newabbrev{\SM}{\emph{Standard Model} (SM)}[SM]


\newabbrev{\LAr}{liquid argon (LAr)}[LAr]
\newabbrev{\SCT}{\emph{Semiconductor Tracker} (SCT)}[SCT]
\newabbrev{\TRT}{\emph{Transition Radiation Tracker} (TRT)}[TRT]
\newabbrev{\TCal}{\emph{Tile Calorimeter} (TileCal)}[TileCal]
\newabbrev{\RoI}{\emph{Region-of-Interest} (RoI)}[RoI]


\newabbrev{\MC}{\emph{Monte Carlo} (MC)}[MC]
\newabbrev{\PFOs}{\emph{Particle Flow Objects} (PFOs)}[PFOs]
\newabbrev{\PFO}{\emph{Particle Flow Object} (PFO)}[PFO]
\newabbrev{\PFa}{\emph{Particle Flow} algorithm (PF algorithm)}[PF algorithm]

\newcommand{\spinhalf}{spin-$\frac{1}{2}$ }
\newcommand\eu{\mathrm{e}}
\newcommand\pT{p_\text{T}}
\newcommand{\red}[1]{\textcolor{red}{#1}}
\newcommand{\green}[1]{\textcolor{green}{#1}}
\renewcommand{\epsilon}{\varepsilon}

\renewcommand{\algorithmicrequire}{\textbf{Input:}}
\renewcommand{\algorithmicensure}{\textbf{Output:}}



