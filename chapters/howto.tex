\chapter{Using CoolThesisSoftware}

Use this appendix to tell the readers (specifically the reviewer) how to use your software. A very reduced example follows; expand as necessary. Description of the program usage (e.g., how to process some example data) should be included as well.

To compile and run the software, you need dependencies XXX and YYY and a C compiler. On Debian-based Linux systems (such as Ubuntu), you may install these dependencies with APT:
\begin{Verbatim}
apt-get install \
  libsuperdependency-dev \
  libanotherdependency-dev \
  build-essential
\end{Verbatim}

To unpack and compile the software, proceed as follows:
\begin{Verbatim}
unzip coolsoft.zip
cd coolsoft
./configure
make
\end{Verbatim}

The program can be used as a C++ library, the simplest use is demonstrated in \cref{lst:ex}. A demonstration program that processes demonstration data is available in directory \verb|demo/|, you can run the program on a demonstration dataset as follows:
\begin{Verbatim}
cd demo/
./bin/cool_process_data data/demo1
\end{Verbatim}

After the program starts, control the data avenger with standard \verb-WSAD- controls.

\begin{listing}
\begin{lstlisting}
#include <CoolSoft.h>
#include <iostream>

int main() {
	int i;
	if(i = cool::ProcessAllData()) // returns 0 on error
		std::cout << i << std::endl;
	else
		std::cerr << "error!" << std::endl;
	return 0;
}
\end{lstlisting}
\caption{Example program.}
\label{lst:ex}
\end{listing}
