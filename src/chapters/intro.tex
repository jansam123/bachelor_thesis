
\chapwithtoc{Introduction}

Particle physics is a branch of physics that studies the fundamental constituents of matter and their interactions.
For physicists to understand the fundamental processes that govern physical reality, we must identify all the participants in experiments that probe them. 
\emph{In this thesis, we focus on tagging jets initiated by quarks or gluons.}
To do the identification, theory, experimental devices, and measured data must be understood in the context of particle identification.

Two fundamental particles that we are interested in are \emph{quarks} and \emph{gluons}~\cite{quarks}, together called \emph{partons}.
The physical theory that describes them is \QCD, which is inherently extremely complicated, far more than the theory of \emph{Electromagnetism}.
The less energetic partons are, the stronger the interactions between them are. 
This creates a problem in describing them as they tend to radiate and split into more partons, which makes it hard to trace them back to their origin.
They create a spray of particles called \emph{jets}.
We discuss the physics of partons in more detail in \cref{ch:physics}.

\green{To produce} these jets (and many others), the \CERN~\cite{cern} built the \LHC~\cite{LHC}, which is the biggest particle accelerator in the world.
On the \LHC is the \emph{ATLAS}~\cite{ATLAS} detector, a giant experiment testing our knowledge of elementary interactions. 
The \LHC accelerates two proton beams in different directions, so they can collide in the ATLAS detector and create a variety of particles.
The ATLAS detector, a massive machine with many subsystems, then detects these particles.
The two main components crucial to our study are the \emph{tracker} and the \emph{calorimeters}.
They provide enough data to reconstruct the jets and their constituents.
A more detailed description of the ATLAS detector is in \cref{ch:atlas}.

The definition of a jet is very complicated, not trivial.
It depends on many aspects of the detector, collisions, algorithm used, and background, secondary collisions called pileup.
Our study uses the anti-$k_t$ algorithm~\cite{antikt} that clusters constituents into jets.
Constituents are energy deposits in the calorimeters or tracks in the tracker if they are charged.
The algorithm that clusters energy deposits into constituents we use is the \PFa algorithm~\cite{PFO}.
It provides the essential data for the jet tagging algorithms and training input of our taggers.
The data we will utilize for the training and testing our tagger is the \MC simulations.
They include the physical simulation of the collision and the detector response.
In \cref{ch:data}, we provide more details about the data we use.

Our goal, telling apart quark/gluon jets, is non-trivial as traces left by these particles in the detector are very similar.
In the past, hand-crafted variables (such as a number of hadron tracks detected in the jet) describing the jets were used~\cite{ntrk_tag}.
Nowadays, modern \ml techniques are used to utilize more variables.
The most adopted and successful is \bdt algorithm~\cite{bdt}, which is still being developed~\cite{bdt_tag}.
It uses multiple \emph{high-level jet variables} (variables that describe the jet as a whole) as input.

In this thesis, we approach the problem using modern \dl techniques.
\dl has exploded in the last decade.
From the \nlp~\cite{bert} to image recognition~\cite{deit3}, \dl has been used to solve many problems.
Most notably, the GPT \cite{gpt3,gpt4} models shook the whole world with their capabilities of text generation, as they can solve complex problems and even supersede humans. 
Or the image-generating models such as Stable Diffusion \cite{stable_diff}, or Dall-E \cite{dalle} that can generate any image from text prompt. \green{[deleted:, making graphics designers obsolete.]}
In this thesis, we utilize the \dl, used in these world-changing models, to solve the problem of quark/gluon tagging.
An in-depth explanation of different \dl techniques and models is in \cref{ch:models}.

Different architectures are explored.
Firstly the \highway architecture~\cite{highway} is developed, which uses high-level jet variables, but considerably more than \bdt. 
Afterward, we turn to a more granular jet description, using \emph{jet constituents}. 
The \EFN and \PFN \cite{efn} are simple architectures that use the jet constituents as an input developed by the \HEP community.

'Attention is all you need'~\cite{att_is_all} introduces a highly successful and influential architecture, called \trans, used in all kinds of applications, most notably \nlp~\cite{bert,gpt3,gpt4,deit3,cait}.
It can learn long-range dependencies and generalize its knowledge to new tasks.
In \HEP community, its use is just starting.
For example, in \cite{qcd_as_nlp}, it is used to teach the model jet structure as a \nlp task.  
Or recently, it was explored by the \CMS~\cite{cms} Collaboration, where they were tagging jets into various classes~\cite{part}.
We improve on their work and apply it specifically to the problem of quark/gluon tagging.
The models are trained, studied comprehensively, and their performance is compared with both \ml and \dl models in \cref{ch:training}.

\green{Quark-gluon taggers are used in many applications, such as the search for new physics and precise measurements of the Standard Model. 
Moreover, the techniques developed in this thesis can be used in other tagging problems, such as the tagging of jets into different classes~\cite{part}.}