
\chapwithtoc{Introduction}

Particle physics is a branch of physics that studies the fundamental constituents of matter and their interactions.
To understand the fundamental processes that govern physical reality, we must identify all the participants in experiments that probe them. 

\emph{In this thesis, we focus on tagging jets initiated by quarks or gluons.}
To do the identification, theory, experimental devices, and measured data must be understood in the context of particle identification.

Two fundamental particles that we are mainly interested in are \emph{quarks} and \emph{gluons}~\cite{quarks}, together called \emph{partons}.
The physical theory that describes them is \QCD, which is inherently extremely complicated, far more than the theory of \emph{Electromagnetism}.
The less energetic partons are, the stronger the interactions between them are. 
This creates a problem in their description as they tend to radiate and split into more partons, which makes it hard to trace them back to their origin.
The developed spray of particles is called \emph{jets}.
We discuss the physics of partons in more detail in \cref{ch:physics}.

To produce these jets (and many other particles), the \CERN~\cite{cern} built the \LHC~\cite{LHC}, which is the biggest particle accelerator in the world.
On the \LHC is the \emph{ATLAS}~\cite{ATLAS} detector, a giant experiment testing our knowledge of elementary interactions. 
The \LHC accelerates two proton beams in different directions, so they can collide at the interaction points and create a variety of particles.
The ATLAS detector, located at one of these interaction points, then detects these particles.
The two main components of ATLAS crucial to our study are the \emph{tracker} and the \emph{calorimeters}.
They provide enough data to reconstruct the jets and their constituents.
A more detailed description of the ATLAS detector is in \cref{ch:atlas}.

The definition of a jet is very complicated.
It depends on many aspects: the detector, collisions, jet algorithm, background events, secondary collisions (pileup), and many more.
Our study uses the anti-$k_t$ algorithm~\cite{antikt} that clusters several constituents into a jet.
The constituent is an energy deposit in the calorimeter or, if it is charged, a track in the tracker.
The algorithm that clusters energy deposits and tracks into constituents we use is the \PFa~\cite{PFO}.
It provides essential input to the jet algorithms and our taggers.

However, to train the neural network we also need the truth information, which cannot be measured but is rather simulated with the \MC simulations.
The \MC includes both the physical simulation of the collision and the detector response.
In \cref{ch:data}, we provide more detail.

Quark-gluon taggers could be used in a variety of applications, such as the search for new physics and precise measurements of the Standard Model. 
To give a concrete example, the detector responses to quarks and gluons differ, so a quark/gluon tagger could be used to do separate calibrations for the two.
In \emph{Vector Boson Fusion} or \emph{Vector Boson Scattering} analyses a well-performing quark tagger would allow for a better event selection.
The SUSY gluino multijet search \cite{susy} could benefit from a quark-gluon tagger.
Moreover, the techniques developed in this thesis can be used in other tagging problems, such as the tagging of jets into different classes~\cite{part}.

Our goal, telling apart quark/gluon jets, is non-trivial as traces left by these particles in the detector are very similar.
In the past, hand-crafted variables (such as a number of hadron tracks detected in the jet) describing the jets were used~\cite{ntrk_tag}.
Nowadays, modern \ml techniques allow the utilization of more variables.
The most adopted and successful is the \bdt algorithm~\cite{bdt}, which is still being developed~\cite{bdt_tag}.
It uses multiple \emph{high-level jet variables} (variables that describe the jet as a whole) as input.

In this thesis, we approach the problem using modern \dl techniques.
\dl has exploded in the last decade.
From the \nlp~\cite{bert} to image recognition~\cite{deit3}, \dl has been used to solve many problems.
Most notably, the GPT models \cite{gpt3,gpt4} shook the whole world with their capabilities of text generation, as they can solve complex problems and even supersede humans in exams. 
Or the image-generating models such as Stable Diffusion \cite{stable_diff}, or Dall-E \cite{dalle} that can generate any image from a text prompt.
We utilize the \dl, used in these world-changing models, to solve the problem of quark/gluon tagging.
An in-depth explanation of different \dl techniques and models is in \cref{ch:models}.

Different architectures are explored.
Firstly the \fc and \highway architectures~\cite{highway,mlp} are developed, which use high-level jet variables, but considerably more than \bdt. 
Afterward, we turn to a more granular jet description, using \emph{jet constituents}. 
The \EFN and \PFN \cite{efn} are simple architectures that use the jet constituents as an input developed by the \HEP community.

'Attention is all you need'~\cite{att_is_all} paper introduces a highly successful and influential architecture, called \trans, used in all kinds of applications, most notably \nlp~\cite{bert,gpt3,gpt4,deit3,cait}.
It can learn long-range dependencies and generalize its knowledge to new tasks.
In \HEP community, its use is just starting.
For example, in \cite{qcd_as_nlp}, it is used to teach the model a jet structure as a \nlp task.  
Or recently, it was explored by the \CMS~\cite{cms} Collaboration, where they were tagging jets into various classes~\cite{part}.
We improve on their work and apply it specifically to the problem of quark/gluon tagging.
The models are trained, studied comprehensively, and their performance is compared in \cref{ch:training}.

